\begin{table}[H]
    \centering
    \small
    \tabcolsep 3pt
    \caption{非文法的な文を判断する精度の指標}
    \label{tab:sample2}
    \begin{tabular}{lcccccc}
    \hline
    \multicolumn{1}{l}{} & \multicolumn{3}{c}{教師役} & \multicolumn{3}{c}{通常利用}\\
    構文種 & 再現率 & 適合率 &F値 & 再現率 & 適合率 &F値\\
    \hline
    受動態 & 1.000 & 0.632 & 0.774 & 1.000 & 0.545 & 0.706 \\
    中間態 & 0.762 & 0.762 & 0.762 & 0.905 & 0.760 & 0.826 \\
    tough構文 & 0.769 & 0.741 & 0.755 & 0.769 & 0.645 & 0.702 \\
    二重目的語構文 & 1.000 & 0.526 & 0.690 & 1.000 & 0.714 & 0.833 \\
    同族目的語 & 0.850 & 0.607 & 0.708 & 0.600 & 0.632 & 0.615 \\
    存在構文 & 0.870 & 0.417 & 0.563 & 0.609 & 0.467 & 0.528 \\
    場所句倒置 & 0.667 & 0.512 & 0.579 & 0.576 & 0.655 & 0.613 \\
    否定倒置(全文/構成素否定) & 0.455 & 0.294 & 0.357 & 0.636 & 0.350 & 0.452 \\
    主語・補語倒置 & 0.667 & 0.500 & 0.571 & 1.000 & 0.563 & 0.720 \\
    命令文 & 0.714 & 0.714 & 0.714 & 0.714 & 0.682 & 0.698 \\
    感嘆文 & 0.650 & 0.448 & 0.531 & 0.550 & 0.393 & 0.458 \\
    疑問文 & 0.852 & 0.575 & 0.687 & 0.926 & 0.556 & 0.694 \\
    多重wh疑問文 & 0.839 & 0.565 & 0.675 & 0.839 & 0.500 & 0.627 \\
    付加疑問文 & 0.906 & 0.744 & 0.817 & 0.938 & 0.833 & 0.882 \\
    否定文と否定極性表現 & 0.563 & 0.243 & 0.340 & 0.563 & 0.273 & 0.367 \\
    強調構文 & 0.800 & 0.632 & 0.706 & 1.000 & 0.500 & 0.667 \\
    擬似分裂文 & 0.700 & 0.500 & 0.583 & 0.800 & 0.471 & 0.593 \\
    繰上げ構文 & 1.000 & 0.364 & 0.533 & 1.000 & 0.348 & 0.516 \\
    話題化構文 & 0.524 & 0.688 & 0.595 & 0.857 & 0.474 & 0.610 \\
    転位構文 & 0.333 & 0.083 & 0.133 & 0.667 & 0.087 & 0.154 \\
    小節 & 0.609 & 0.500 & 0.549 & 0.609 & 0.452 & 0.519 \\
    that節 & 0.691 & 0.623 & 0.655 & 0.727 & 0.606 & 0.661 \\
    間接疑問文 & 0.900 & 0.783 & 0.837 & 0.800 & 0.800 & 0.800 \\
    関係節 & 0.648 & 0.556 & 0.598 & 0.611 & 0.559 & 0.584 \\
    比較構文 & 0.618 & 0.538 & 0.575 & 0.559 & 0.731 & 0.633 \\
    副詞節 & 0.842 & 0.457 & 0.593 & 0.895 & 0.415 & 0.567 \\
    譲歩節 & 0.634 & 0.765 & 0.693 & 0.585 & 0.774 & 0.667 \\
    挿入節 & 0.842 & 0.552 & 0.667 & 0.789 & 0.484 & 0.600 \\
    仮定法 & 0.696 & 0.667 & 0.681 & 0.870 & 0.588 & 0.702 \\
    等位構造 & 0.800 & 0.583 & 0.675 & 0.771 & 0.628 & 0.692 \\
    外置構文 & 0.862 & 0.532 & 0.658 & 0.690 & 0.500 & 0.580 \\
    重名詞句転位 & 0.913 & 0.538 & 0.677 & 0.783 & 0.621 & 0.692 \\
    二次述語 & 1.000 & 0.318 & 0.483 & 1.000 & 0.500 & 0.667 \\
    寄生空所 & 0.654 & 0.567 & 0.607 & 0.423 & 0.579 & 0.489 \\
    省略文 & 0.289 & 0.786 & 0.423 & 0.368 & 0.700 & 0.483 \\
    不定詞節 & 0.800 & 0.923 & 0.857 & 0.800 & 0.923 & 0.857 \\
    動名詞節 & 0.722 & 0.765 & 0.743 & 0.639 & 0.821 & 0.719 \\
    分詞構文 & 1.000 & 0.444 & 0.615 & 1.000 & 0.400 & 0.571 \\
    属格化 & 0.304 & 0.700 & 0.424 & 0.304 & 0.438 & 0.359 \\
    (非)能格・非対格動詞 & 0.897 & 0.946 & 0.921 & 0.846 & 0.767 & 0.805 \\
    知覚動詞 & 0.514 & 0.633 & 0.567 & 0.514 & 1.000 & 0.679 \\
    使役動詞 & 0.324 & 0.571 & 0.414 & 0.324 & 0.600 & 0.421 \\
    叙実動詞 & 0.478 & 0.733 & 0.579 & 0.565 & 0.765 & 0.650 \\
    心理動詞 & 0.407 & 0.786 & 0.537 & 0.704 & 0.826 & 0.760 \\
    場所格交替動詞 & 0.600 & 0.667 & 0.632 & 0.600 & 0.545 & 0.571 \\
    句動詞 & 0.516 & 0.800 & 0.627 & 0.516 & 0.800 & 0.627 \\
    成句 & - & 0.000 & - & - & 0.000 & - \\
    (法)助動詞 & 0.700 & 0.636 & 0.667 & 0.800 & 0.533 & 0.640 \\
    準助動詞 & 0.286 & 0.667 & 0.400 & 0.429 & 0.750 & 0.545 \\
    相 & 0.516 & 0.842 & 0.640 & 0.516 & 0.696 & 0.593 \\
    副詞 & 0.167 & 1.000 & 0.286 & 0.121 & 1.000 & 0.216 \\
    (再帰)代名詞 & 0.263 & 0.714 & 0.385 & 0.263 & 0.625 & 0.370 \\
    数量詞作用域 & 0.429 & 0.750 & 0.545 & 0.714 & 0.556 & 0.625 \\
    数量詞遊離 & 0.686 & 0.778 & 0.729 & 0.784 & 0.833 & 0.808 \\
    句構造 & 0.500 & 0.714 & 0.588 & 0.600 & 0.667 & 0.632 \\
    優位性効果・複合名詞句等の島制約 & 0.531 & 0.722 & 0.612 & 0.551 & 0.844 & 0.667 \\
    \hline
    \end{tabular}
\end{table}
