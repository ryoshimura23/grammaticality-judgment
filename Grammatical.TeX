\begin{table}[H]
    \centering
    \small
    \tabcolsep 3pt
    \caption{文法的な文を判断する精度の指標}
    \label{tab:sample2}
    \begin{tabular}{lcccccc}
    \hline
    \multicolumn{1}{l}{} & \multicolumn{3}{c}{教師役} & \multicolumn{3}{c}{通常利用}\\
    構文種 & 再現率 & 適合率 &F値 & 再現率 & 適合率 &F値\\
    \hline
    受動態 & 0.825 & 1.000 & 0.904 & 0.750 & 1.000 & 0.857 \\ 
    中間態 & 0.815 & 0.815 & 0.815 & 0.778 & 0.913 & 0.840 \\ 
    tough構文 & 0.896 & 0.909 & 0.902 & 0.836 & 0.903 & 0.868 \\ 
    二重目的語構文 & 0.591 & 1.000 & 0.743 & 0.818 & 1.000 & 0.900 \\ 
    同族目的語 & 0.703 & 0.897 & 0.788 & 0.811 & 0.789 & 0.800 \\ 
    存在構文 & 0.462 & 0.889 & 0.608 & 0.692 & 0.800 & 0.742 \\ 
    場所句倒置 & 0.740 & 0.851 & 0.792 & 0.870 & 0.827 & 0.848 \\ 
    否定倒置(全文/構成素否定) & 0.782 & 0.878 & 0.827 & 0.764 & 0.913 & 0.832 \\ 
    主語・補語倒置 & 0.760 & 0.864 & 0.809 & 0.720 & 1.000 & 0.837 \\ 
    命令文 & 0.786 & 0.786 & 0.786 & 0.750 & 0.778 & 0.764 \\ 
    感嘆文 & 0.775 & 0.887 & 0.827 & 0.761 & 0.857 & 0.806 \\ 
    疑問文 & 0.750 & 0.927 & 0.829 & 0.706 & 0.960 & 0.814 \\ 
    多重wh疑問文 & 0.655 & 0.884 & 0.752 & 0.552 & 0.865 & 0.674 \\ 
    付加疑問文 & 0.783 & 0.923 & 0.847 & 0.870 & 0.952 & 0.909 \\ 
    否定文と否定極性表現 & 0.682 & 0.896 & 0.774 & 0.727 & 0.901 & 0.805 \\ 
    強調構文 & 0.816 & 0.912 & 0.861 & 0.605 & 1.000 & 0.754 \\ 
    擬似分裂文 & 0.767 & 0.885 & 0.821 & 0.700 & 0.913 & 0.792 \\
    繰上げ構文 & 0.736 & 1.000 & 0.848 & 0.717 & 1.000 & 0.835 \\
    話題化構文 & 0.828 & 0.706 & 0.762 & 0.310 & 0.750 & 0.439 \\
    転位構文 & 0.633 & 0.905 & 0.745 & 0.300 & 0.900 & 0.450 \\
    小節 & 0.774 & 0.842 & 0.807 & 0.726 & 0.833 & 0.776 \\
    that節 & 0.733 & 0.788 & 0.759 & 0.698 & 0.800 & 0.745 \\
    間接疑問文 & 0.886 & 0.951 & 0.918 & 0.909 & 0.909 & 0.909 \\
    関係節 & 0.813 & 0.865 & 0.838 & 0.827 & 0.855 & 0.841 \\
    比較構文 & 0.719 & 0.780 & 0.748 & 0.891 & 0.792 & 0.838 \\
    副詞節 & 0.725 & 0.943 & 0.820 & 0.652 & 0.957 & 0.776 \\
    譲歩節 & 0.895 & 0.819 & 0.855 & 0.908 & 0.802 & 0.852 \\
    挿入節 & 0.764 & 0.933 & 0.840 & 0.709 & 0.907 & 0.796 \\
    仮定法 & 0.887 & 0.900 & 0.894 & 0.803 & 0.950 & 0.870 \\
    等位構造 & 0.701 & 0.870 & 0.777 & 0.761 & 0.864 & 0.810 \\
    外置構文 & 0.560 & 0.875 & 0.683 & 0.600 & 0.769 & 0.674 \\
    重名詞句転位 & 0.526 & 0.909 & 0.667 & 0.711 & 0.844 & 0.771 \\
    二次述語 & 0.559 & 1.000 & 0.717 & 0.794 & 1.000 & 0.885 \\
    寄生空所 & 0.606 & 0.690 & 0.645 & 0.758 & 0.625 & 0.685 \\
    省略文 & 0.932 & 0.603 & 0.732 & 0.864 & 0.613 & 0.717 \\
    不定詞節 & 0.972 & 0.921 & 0.946 & 0.972 & 0.921 & 0.946 \\
    動名詞節 & 0.822 & 0.787 & 0.804 & 0.889 & 0.755 & 0.816 \\
    分詞構文 & 0.873 & 1.000 & 0.932 & 0.848 & 1.000 & 0.918 \\
    属格化 & 0.941 & 0.750 & 0.835 & 0.824 & 0.724 & 0.771 \\
    (非)能格・非対格動詞 & 0.969 & 0.940 & 0.955 & 0.846 & 0.902 & 0.873 \\
    知覚動詞 & 0.982 & 0.875 & 0.926 & 1.000 & 0.760 & 0.864 \\
    使役動詞 & 0.809 & 0.603 & 0.691 & 0.830 & 0.609 & 0.703 \\
    叙実動詞 & 0.939 & 0.756 & 0.838 & 0.879 & 0.744 & 0.806 \\
    心理動詞 & 0.959 & 0.814 & 0.881 & 0.945 & 0.896 & 0.920 \\
    場所格交替動詞 & 0.947 & 0.931 & 0.939 & 0.912 & 0.929 & 0.920 \\
    句動詞 & 0.929 & 0.776 & 0.846 & 0.946 & 0.791 & 0.862 \\
    成句 & 0.957 & 1.000 & 0.978 & 0.957 & 1.000 & 0.978 \\
    (法)助動詞 & 0.925 & 0.942 & 0.933 & 0.868 & 0.958 & 0.911 \\
    準助動詞 & 0.944 & 0.773 & 0.850 & 0.944 & 0.810 & 0.872 \\
    相 & 0.948 & 0.786 & 0.859 & 0.879 & 0.773 & 0.823 \\
    副詞 & 1.000 & 0.590 & 0.742 & 1.000 & 0.577 & 0.731 \\
    (再帰)代名詞 & 0.933 & 0.667 & 0.778 & 0.900 & 0.659 & 0.761 \\
    数量詞作用域 & 0.971 & 0.895 & 0.932 & 0.886 & 0.939 & 0.912 \\
    数量詞遊離 & 0.878 & 0.818 & 0.847 & 0.902 & 0.871 & 0.886 \\
    句構造 & 0.966 & 0.918 & 0.941 & 0.948 & 0.932 & 0.940 \\
    優位性効果・複合名詞句等の島制約 & 0.791 & 0.607 & 0.687 & 0.930 & 0.667 & 0.777 \\
    \hline
    \end{tabular}
\end{table}
